%
% ---------------------------------------------------------------
% Copyright (C) 2012-2018 Gang Li
% ---------------------------------------------------------------
%
% This work is the default powerdot-tuliplab style test file and may be
% distributed and/or modified under the conditions of the LaTeX Project Public
% License, either version 1.3 of this license or (at your option) any later
% version. The latest version of this license is in
% http://www.latex-project.org/lppl.txt and version 1.3 or later is part of all
% distributions of LaTeX version 2003/12/01 or later.
%
% This work has the LPPL maintenance status "maintained".
%
% This Current Maintainer of this work is Gang Li.
%
%

\documentclass[
 size=14pt,
 paper=smartboard,  %a4paper, smartboard, screen
 mode=present, 		%present, handout, print
 display=slides, 	% slidesnotes, notes, slides
 style=tuliplab,  	% TULIP Lab style
 pauseslide,
 fleqn,leqno]{powerdot}


\usepackage{cancel}
\usepackage{caption}
\usepackage{stackengine}
\usepackage{smartdiagram}
\usepackage{attrib}
\usepackage{amssymb}
\usepackage{amsmath} 
\usepackage{amsthm} 
\usepackage{mathtools}
\usepackage{rotating}
\usepackage{graphicx}
\usepackage{boxedminipage}
\usepackage{rotate}
\usepackage{calc}
\usepackage[absolute]{textpos}
\usepackage{psfrag,overpic}
\usepackage{fouriernc}
\usepackage{pstricks,pst-3d,pst-grad,pstricks-add,pst-text,pst-node,pst-tree}
\usepackage{moreverb,epsfig,subfigure}
\usepackage{color}
\usepackage{booktabs}
\usepackage{etex}
\usepackage{breqn}
\usepackage{multirow}
\usepackage{natbib}
\usepackage{bibentry}
\usepackage{gitinfo2}
\usepackage{siunitx}
\usepackage{nicefrac}
%\usepackage{geometry}
%\geometry{verbose,letterpaper}
\usepackage{media9}
\usepackage{animate}
%\usepackage{movie15}
\usepackage{auto-pst-pdf}

\usepackage{breakurl}
\usepackage{fontawesome}
\usepackage{xcolor}
\usepackage{multicol}


\usepackage{mdwlist} 
\usepackage{verbatim}
\usepackage[utf8]{inputenc}
%\usepackage{dtk-logos}
\usepackage{C:/texlive/2022/texmf-dist/tex/latex/dtk/dtk-logos}
\usepackage{tikz}
\usepackage{adigraph}
%\usepackage{tkz-graph}
\usepackage{hyperref}
%\usepackage{ulem}
\usepackage{pgfplots}
\usepackage{verbatim}
\usepackage{fontawesome}


\usepackage{todonotes}
% \usepackage{pst-rel-points}
\usepackage{animate}
\usepackage{fontawesome}
\usepackage{makecell}
\usepackage{listings}
\lstset{frameround=fttt,
frame=trBL,
stringstyle=\ttfamily,
backgroundcolor=\color{yellow!20},
basicstyle=\footnotesize\ttfamily}
\lstnewenvironment{code}{
\lstset{frame=single,escapeinside=`',
backgroundcolor=\color{yellow!20},
basicstyle=\footnotesize\ttfamily}
}{}


\usepackage{hyperref}
\hypersetup{ % TODO: PDF meta Data
  pdftitle={Presentation Title},
  pdfauthor={Gang Li},
  pdfpagemode={FullScreen},
  pdfborder={0 0 0}
}


% \usepackage{auto-pst-pdf}
% package to show source code

\definecolor{LightGray}{rgb}{0.9,0.9,0.9}
\newlength{\pixel}\setlength\pixel{0.000714285714\slidewidth}
\setlength{\TPHorizModule}{\slidewidth}
\setlength{\TPVertModule}{\slideheight}
\newcommand\highlight[1]{\fbox{#1}}
\newcommand\icite[1]{{\footnotesize [#1]}}

\newcommand\twotonebox[2]{\fcolorbox{pdcolor2}{pdcolor2}
{#1\vphantom{#2}}\fcolorbox{pdcolor2}{white}{#2\vphantom{#1}}}
\newcommand\twotoneboxo[2]{\fcolorbox{pdcolor2}{pdcolor2}
{#1}\fcolorbox{pdcolor2}{white}{#2}}
\newcommand\vpspace[1]{\vphantom{\vspace{#1}}}
\newcommand\hpspace[1]{\hphantom{\hspace{#1}}}
\newcommand\COMMENT[1]{}

\newcommand\placepos[3]{\hbox to\z@{\kern#1
        \raisebox{-#2}[\z@][\z@]{#3}\hss}\ignorespaces}

\renewcommand{\baselinestretch}{1.2}


\newcommand{\draftnote}[3]{
	\todo[author=#2,color=#1!30,size=\footnotesize]{\textsf{#3}}	}
% TODO: add yourself here:
%
\newcommand{\gangli}[1]{\draftnote{blue}{GLi:}{#1}}
\newcommand{\shaoni}[1]{\draftnote{green}{sn:}{#1}}
\newcommand{\gliMarker}
	{\todo[author=GLi,size=\tiny,inline,color=blue!40]
	{Gang Li has worked up to here.}}
\newcommand{\snMarker}
	{\todo[author=Sn,size=\tiny,inline,color=green!40]
	{Shaoni has worked up to here.}}

%%%%%%%%%%%%%%%%%%%%%%%%%%%%%%%%%%%%%%%%%%%%%%%%%%%%%%%%%%%%%%%%%%%%%%%%
% title
% TODO: Customize to your Own Title, Name, Address
%
\title{Titanic: Machine Learning from Disaster}
\author{
Ziqi Lin
\\
\\Nanjing University of Science and Technology
\\Deakin University
}
\date{\gitCommitterDate}


% Customize the setting of slides
\pdsetup{
% TODO: Customize the left footer, and right footer
rf=\href{http://www.tulip.org.au}{
Last Changed by: \textsc{\gitCommitterName}\ \gitVtagn -\gitAbbrevHash\ (\gitAuthorDate)
},
cf={Traffic Flow Prediction In a U.S Metropolis},
}


\begin{document}

\maketitle

%\begin{slide}{Overview}
%\tableofcontents[content=sections]
%\end{slide}


%%==========================================================================================
%%
\begin{slide}[toc=,bm=]{Overview}
\tableofcontents[content=currentsection,type=1]
\end{slide}
%%
%%==========================================================================================


\section{Problem Definition}


%%==========================================================================================
%%
\begin{slide}{Tabular Playground Series - Sep 2021}
\begin{center}

\twotonebox{\rotatebox{90}{Defn}}{\parbox{.86\textwidth}
{The project aims to predict which passengers survived the Titanic sinking
\begin{itemize}
\item On April 15, 1912, there were not enough lifeboats on board, resulting in the deaths of 1,502 of the 2,224 passengers and crew. (Survival rate: 0.3246). 
\item While there is some luck involved in survival, it seems that some groups of people are more likely to survive than others.
\end{itemize}
}}
\end{center}
\begin{table}[htbp]
	\setlength{\abovecaptionskip}{0pt}
	\setlength{\belowcaptionskip}{10pt}
	\centering
	\caption{Data}
\begin{tabular}{ c | c | c }
	\toprule
	Data & Count  & feature   \\
	\midrule
	$train.csv$
	&  {$419$} &  {$12$}  \\
	$test.csv$
	&  {$892$} &  {$12$}      \\
	\bottomrule	
\end{tabular}
\end{table}

\end{slide}
%%
%%==========================================================================================
\section{Data Analysis}
%%==========================================================================================
%%
\begin{slide}{Data Description}
\begin{center}
\begin{table}[htbp]
	\setlength{\abovecaptionskip}{0pt}
	\setlength{\belowcaptionskip}{10pt}
	\centering
	\caption{Data}
	\begin{tabular}{ c | c }
		\toprule
		feature & feature   \\
		\midrule
		$PassengerId$
		&  {Survived}  \\
		$Pclass$
		&  {Name}      \\
        $Sex$
        &  {Age}  \\
        $SibSp$
        &  {Parch}  \\
        $Ticket$
        &  {Fare}  \\
        $Cabin$
        &  {Embarked}  \\
  
		\bottomrule	
	\end{tabular}
\end{table}
\end{center}

\end{slide}
%%
%%==========================================================================================
\begin{slide}{Data Description}
\begin{center}

\twotonebox{\rotatebox{90}{Defn}}{\parbox{.86\textwidth}
{Based on the information in the data, the following conclusions can be drawn:
\begin{itemize}
\item Character Name: Although it is a string, it contains some gender characteristics, such as (Mr, Mrs, Miss).
\item Feature Ticket: contains a combination of numbers and letters, so it is difficult to find some patterns.
\item Feature Cabin: There are missing values in the cabin. We need to check further to see if there are also missing values in the cabin.
\end{itemize}
}}
\end{center}

\end{slide}
%%Data structure
%%==========================================================================================
\begin{slide}{Data structure}
\begin{table}[htbp]
	\setlength{\abovecaptionskip}{0pt}
	\setlength{\belowcaptionskip}{10pt}
	\centering
	\caption{Number of missing features}
\begin{tabular}{ c | c | c }
	\toprule
	Data & feature  &  count  \\
	\midrule
	$train.csv$
	&  {$Age$} &  {$0.1987$}  \\
    &  {$Cabin$} &  {$0.771$}  \\
    &  {$Embarked$} &  {$0.0022$}  \\
	$test.csv$
	&  {$Age$} &  {$0.2057$}      \\
    &  {$Cabin$} &  {$0.6778$}      \\
    &  {$Fare$} &  {$0.0021$}      \\
	\bottomrule	
\end{tabular}
\end{table}

\end{slide}
%%111
%%==========================================================================================
\begin{slide}{The correlation of a single feature to a tag}
\begin{center}
\begin{figure}
	\setlength{\abovecaptionskip}{0.5cm}
    \centering
    \includegraphics[scale=0.5]{C:/Users/77967/Desktop/Flip00/2.eps}\\	
    \caption{Correlations between individual features and survival}
    \label{fig:Correlations between individual features and survival}
\end{figure}
\end{center}

\end{slide}
%%summarize
%%==========================================================================================
\begin{slide}{Summarize}
\begin{center}

\twotonebox{\rotatebox{90}{Defn}}{\parbox{.86\textwidth}
{Data analysis and summary:
\begin{itemize}
\item PassengerId is ID, Ticket feature is difficult to find regularity, Cabin is too much missing, so it needs to be removed.
\item The Name feature can extract part of gender, married and unmarried information, identity, etc. (The Name feature itself needs to be deleted).
\item Sex, Pclass, Age, Embarked on, Fare are very important features, and a certain relationship is established between them.
\item The missing data in Age, Embarked and Fare need to be embarked on.
\end{itemize}
}}
\end{center}

\end{slide}
%%summarize
%%==========================================================================================
\section{Data Processing}

%%==========================================================================================


\begin{slide}{Data Processing}

\begin{center}
\begin{itemize}

\item
\smallskip
\large
{Delete invalid feature\\
After the above analysis, Passengerld, Ticket and Cabin features were removed.\\
}
\item
\smallskip
\large
{Extract valid features\\
The Name feature, usually preceded by ".", represents an identity and can be extracted using a regular expression.\\
}
\item
\smallskip
\large
{Transformation feature\\
Since gender is represented as a string, we can represent it numerically.\\
}

\end{itemize}
\end{center}

\end{slide}

%%==========================================================================================


\begin{slide}{Missing value padding}

\begin{center}
\begin{itemize}

\item
\smallskip
\large
{1.Age\\
Select the random selection of mean and standard deviation of Age under Pclass and Gender matching.\\
}
\item
\smallskip
\large
{2.Embarked\\
There are only two missing values in the training set, so we take the simplest most commonly used padding and convert it into a classified numerical feature.\\
}
\item
\smallskip
\large
{3.Fare\\
One sample in the test set is missing the Fare feature, and we simply fill it with the median.\\
}

\end{itemize}
\end{center}

\end{slide}

%%==========================================================================================

%%

\section{Modelling}
%%==========================================================================================
%%
\begin{slide}{Model Train}
	
\begin{center}
\begin{itemize}

\item
\smallskip
\large
{Train\\
Model: Logistic Regression\\

Mode2: Decision Tree\\

Mode3: KNN\\

Mode4: Naive Bayes Classifier\\

Mode5: SVM\\

Mode6: MLP\\
}

\end{itemize}
\end{center}
\end{slide}

%%==========================================================================================
\begin{slide}{Train results}
\begin{table}[htbp]
\setlength{\abovecaptionskip}{0pt}
\setlength{\belowcaptionskip}{10pt}
\centering
\caption{Train results}
\begin{tabular}{ c | c }
\toprule
Algorithm & Score  \\
\midrule
$Support Vector Machines$
&  {$0.827142$}\\
$Decision Tree$
&  {$0.824920$}\\
$MLP$
&  {$0.812592$}\\
$KNN$
&  {$0.811437$}\\
$Logistic Regression$
&  {$0.806955$}\\
$Naive Bayes $
&  {$0.783435$}\\
\bottomrule	
\end{tabular}
\end{table}
\end{slide}
%%==========================================================================================

%%

\section{Prediction}

%%==========================================================================================
%%
\begin{slide}[toc=,bm=]{Make Prediction}

\begin{table}[htbp]
\setlength{\abovecaptionskip}{1pt}
\setlength{\belowcaptionskip}{3pt}
\centering
\caption{Partial test results}
\begin{tabular}{ c | c }
\toprule
PassengerId & Survived  \\
\midrule
$892$
&  {$1$}\\
$893$
&  {$1$}\\
$894$
&  {$0$}\\
$895$
&  {$1$}\\
$896$
&  {$0$}\\
$897$
&  {$1$}\\
$898$
&  {$0$}\\

\bottomrule	
\end{tabular}
\end{table}
%%==========================================================================================
\end{slide}
%%
%%==========================================================================================

%%==========================================================================================
% TODO: Contact Page
\begin{wideslide}[toc=,bm=]{Contact Information}
\centering
\vspace{\stretch{1}}
\twocolumn[
lcolwidth=0.35\linewidth,
rcolwidth=0.65\linewidth
]
{
% \centerline{\includegraphics[scale=.2]{tulip-logo.eps}}
}
{
\vspace{\stretch{1}}
Graduate student Ziqi Lin\\
School of Economics and Management\\
Nanjing University of Science and Technology, China
\begin{description}
 \item[\textcolor{orange}{\faEnvelope}] \href{mailto:jluo@tulip.org.au}
 {\textsc{\footnotesize{zqlin@tulip.academy}}}

 \item[\textcolor{orange}{\faHome}] \href{http://www.tulip.org.au}
 {\textsc{\footnotesize{Team for Universal Learning and Intelligent Processing}}}
\end{description}
}
\vspace{\stretch{1}}
\end{wideslide}

\end{document}

\endinput
